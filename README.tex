
\documentclass[DIV=calc, paper=letter, fontsize=11pt]{scrartcl}	 % A4 paper and 11pt font size

\usepackage[body={6.5in,9.0in},
  top=1.0in, left=1.0in]{geometry}
  
\usepackage[english]{babel} % English language/hyphenation
\usepackage[protrusion=true,expansion=true]{microtype} % Better typography
\usepackage{amsmath,amsfonts,amsthm} % Math packages
\usepackage[svgnames]{xcolor} % Enabling colors by their 'svgnames'
\usepackage[hang, small,labelfont=bf,up,textfont=it,up]{caption} % Custom captions under/above floats in tables or figures
\usepackage{booktabs} % Horizontal rules in tables
\usepackage{fix-cm}	 % Custom font sizes - used for the initial letter in the document

\usepackage{sectsty} % Enables custom section titles
\allsectionsfont{\usefont{OT1}{phv}{b}{n}} % Change the font of all section commands

\usepackage{fancyhdr} % Needed to define custom headers/footers
\pagestyle{fancy} % Enables the custom headers/footers
\usepackage{lastpage} % Used to determine the number of pages in the document (for "Page X of Total")
%\usepackage[squaren]{mySIunits}
\usepackage{epsfig}

\usepackage{fancyvrb}% used to include files verbatim

\usepackage{hyperref}

\usepackage[backend=bibtex,style=numeric,sorting=ydnt,maxnames=15]{biblatex}
\renewbibmacro{in:}{}
\usepackage{chemsym}

% Count total number of entries in each refsection
\AtDataInput{%
  \csnumgdef{entrycount:\therefsection}{%
    \csuse{entrycount:\therefsection}+1}}

% Print the labelnumber as the total number of entries in the
% current refsection, minus the actual labelnumber, plus one
\DeclareFieldFormat{labelnumber}{\mkbibdesc{#1}}    
\newrobustcmd*{\mkbibdesc}[1]{%
  \number\numexpr\csuse{entrycount:\therefsection}+1-#1\relax}

% Headers - all currently empty
\lhead{}
\chead{}
\rhead{}

% Footers
\lfoot{\textsf{CMU-thesis} manual, v1.1, \today}
\cfoot{}
\rfoot{\footnotesize Page \thepage\ of \pageref{LastPage}} % "Page 1 of 2"

\renewcommand{\headrulewidth}{0.0pt} % No header rule
\renewcommand{\footrulewidth}{0.4pt} % Thin footer rule

\usepackage{lettrine} % Package to accentuate the first letter of the text
\newcommand{\initial}[1]{ % Defines the command and style for the first letter
\lettrine[lines=3,lhang=0.3,nindent=0em]{
\color{DarkGoldenrod}
{\textsf{#1}}}{}}

\usepackage{titling} % Allows custom title configuration

\newcommand{\HorRule}{\color{DarkGoldenrod} \rule{\linewidth}{1pt}} % Defines the gold horizontal rule around the title

\pretitle{\vspace{-1.5in} \begin{center} \HorRule \fontsize{25}{25} \usefont{OT1}{phv}{b}{n} \color{DarkRed} \selectfont} % Horizontal rule before the title

\title{How to use the CMU-thesis \LaTeX\ style} % Your article title

\posttitle{\par\end{center}\vskip 0.5em} % Whitespace under the title

\preauthor{\begin{center}\large \lineskip 0.5em \usefont{OT1}{phv}{b}{sl} \color{DarkRed}} % Author font configuration

\author{\vspace*{-0.7in}} % Your name

\postauthor{\footnotesize \usefont{OT1}{phv}{m}{sl} \color{Black} % Configuration for the institution name

\par\end{center}\HorRule} % Horizontal rule after the title
\date{CMU-thesis Manual, v1.0, \today}

\newcommand{\la}[2]{\textsf{$\backslash$#1}\{#2\}}
\newcommand{\lb}[1]{\textsf{$\backslash$#1}}
\newcommand{\lc}[3]{\textsf{$\backslash$#1}\{#2\}\{#3\}}
\newcommand{\bv}[1]{\mathbf{#1}}
\newcommand{\Css}[2]{C^{#1}_{#2}}
\newcommand{\Cdg}{C$^\circ$}
\newcommand{\beryl}{\Be$_3$\Al$_2$\Si$_6$\O$_{18}$}
\newcommand{\TEM}{Transmission Electron Microscopy}

\begin{document}
\maketitle

\renewcommand{\contentsname}{Table of Contents}
{\small\tableofcontents}


\newpage
\section{Introduction}
\LaTeX\ provides the user with a mature professional typesetting system that can be used to 
produce nice-looking documents while at the same time removing some of the tediousness of dealing
with citations, renumbering things while moving figures around, and so on.  In this document, we will assume that
the reader has at least some familiarity with \LaTeX\footnote{There are many good resources and examples available
on the web and in several books; the most useful reference book would be ``The \LaTeX\ Companion'' (2nd edition) by 
Frank Mittelbach and Michel Goossens, Addison Wesley, 2004.} so that basic document typesetting is not a big mystery.

Since typesetting requires running the \TeX\ program, we will also assume that a fully functional \LaTeX\ implementation
is installed on the user's computer.  There are many good implementations available on the web; for  
Mac OS X, we recommend \TeX Live-2013 along with the \textsf{TeXShop} editor.  For the PC platform, the \textsf{MiKTeX} implementation
is quite good.  Obviously, other editors will work as well (vim, emacs, etc), since the input files are simple text files.
Dealing with citations is much easier if you use one of the freely available programs: on the Mac OS X platform, \textsf{BiBDesk}
is really good, while on the PC you might want to use \textsf{JabRef}.  Note that \textsf{EndNote} can export bibliography
files for use with \LaTeX, so if you already have an extensive citation database in EndNote, it would be trivial to 
convert it to the BiBTeX format.

In the following sections, we describe in some detail how to use the CMU-thesis style.  Briefly, you can use this 
style to produce a complete overview document or a final thesis document, as well as all the announcements sheets.

\section{Obtaining the CMU-thesis package}
All the files needed to use the CMU-thesis package are maintained on the department's source code
repository, \textsf{gitorious.materials.cmu.edu}.  Simply look for the \textsf{origin/LaTeX} branch and fetch
the entire archive or individual files.  Note that this branch is read only; you can fork the branch into your
own account, using the regular pull command.  The files will be maintained by MDG; any problems, changes, 
requests for additions, etc., should be addressed to \textsf{degraef@cmu.edu}.


\section{The CMU-thesis package}
The CMU thesis style is defined in a series of files:
\begin{itemize}
	\item \textsf{cmu-thesis.cls} is the main class file, and defines all the necessary formatting commands; it is loosely based 
	on a similar file from the University of Michigan, but has since (2007) been modified extensively for local use.  There is no need
	to ever edit (or even open) this file, unless you're really curious to know how things work.  
	\item \textsf{preamble.ins} is a file that is called (inserted) by the main class file and contains a number of user-definable commands; see
	section~\ref{sec:preamble} for details.
	\item \textsf{sample-thesis.tex} is an example file that uses the \textsf{cmu-thesis.cls} class file;  details are described in 
	section~\ref{sec:sample}.
	\item \textsf{frontmatter.tex} contains all the code for dealing with overview/thesis announcements, title page, abstract, etc.  See section~\ref{sec:frontmatter}.
	\item \textsf{references.tex} is a file that can be included in the main document; as explained in section~\ref{sec:references}, it provides
	a path to your BiBTeX bibliography files.
	\item \textsf{README.tex} is the \LaTeX\ source for this manual;\footnote{Note that this manual is formatted using
	the \textsf{scrartcl} class which may or may not be installed in your \LaTeX\ distribution.  This class is typically useful for 
	documents that are formatted on the European A4 paper format; given that the author of this document has his roots in
	Europe, it shows that old habits die hard $\ldots$.  If you can not typeset this manual on your system, no worries, simply
	read the pdf file.} the corresponding pdf file is also provided.
\end{itemize}


\subsection{User-definable commands in \textsf{preamble.ins}\label{sec:preamble}}
The \textsf{preamble.ins} file is used to define a number of commands that are specific
to your overview or thesis document. They are placed in a separate file to keep the 
\textsf{main.tex} file uncluttered.  The default preamble file looks as follows:
\fvset{frame=lines,numbers=left,formatcom=\color{blue},fontsize=\footnotesize}
\VerbatimInput{preamble.ins}
Before you start using the \textsf{cmu-thesis} class, you must edit this file and change the
arguments of the commands on lines 6-20 to appropriate arguments.  The command name 
should be clear enough for you to know what is needed as argument.

On line 25, an extra file is read in using the \la{input}{} command. This file can be used to 
store all the user defined commands.  The file is separate from the preamble file so that a new
version of the preamble will not overwrite the already existing user-defined commands. The
\textsf{mycommands.ins} file is not part of the \textsf{cmu-thesis} distribution.

In the \textsf{mycommands.ins} file, you can define your own commands for things that you will
need frequently, such as complex chemical formulae, a long word or word sequence that
you don't want to retype each time, etc...  Here is how to define a short hand notation
for the chemical composition of the mineral beryl:
\begin{verbatim}
  \newcommand{\beryl}{\Be$_3$\Al$_2$\Si$_6$\O$_{18}$}
\end{verbatim}
To use this command in a sentence, you would simply type \lb{beryl}$\backslash$ to obtain \beryl\ in 
the middle of the sentence. Note that the command is terminated by a backslash; the reason for this 
backslash is to force the typesetting engine to leave a space between the symbol and 
the following word.  In other words, without the end backslash we would read \beryl and the next word
(and) would be right up against the symbol; in case you didn't notice it, on the previous line, there is no space
between the subscript $_{18}$ and the word ``and.'' 

Here is how to define a shorthand notation for the ``degrees centigrade'' symbol:
\begin{verbatim}
  \newcommand{\Cdg}{C$^\circ$}
\end{verbatim}
which can be used in a sentence by typing \lb{Cdg}$\backslash$ to get $100$ \Cdg\ or between parenthesis 
as (\Cdg); in this latter case, there is no need for the ending backslash because the closing parenthesis takes on that role.

New commands can also have one or more arguments, as in the next example:
\begin{verbatim}
  \newcommand{\bv}[1]{\mathbf{#1}}	
\end{verbatim}
This is a shorthand notation for bold-faced vector symbols, as in \$\la{bv}{q}\$ to get the vector $\bv{q}$, which would
normally require \$\la{mathbf}{q}\$. Note that the dollar signs
are not part of the definition of \la{bv}{}, so that the same command can also be called from inside displayed equations.
If more than one argument is needed, the same command declaration format can be used:
\begin{verbatim}
  \newcommand{\Css}[2]{C^{#1}_{#2}}	
\end{verbatim}
This command contains math symbols for subscript and superscript, so it must be used inside a math environment;
typing 
\begin{verbatim}
  \begin{equation}
     \Css{\alpha}{\beta} = \sum_{k=1}^{3}\Css{\alpha}{k}\Css{k}{\beta}.\nonumber
   \end{equation}
\end{verbatim}
yields:
\begin{equation}
     \Css{\alpha}{\beta} = \sum_{k=1}^{3}\Css{\alpha}{k}\Css{k}{\beta}.\nonumber
\end{equation}
There are many other commands that the user might want to define; for instance, instead of having to type Transmission Electron Microscopy each time,
we can simply use the command \lb{TEM}, defined as:
\begin{verbatim}
	\newcommand{\TEM}{Transmission Electron Microscopy}
\end{verbatim}
and use it as \lb{TEM}$\backslash$ to obtain \TEM\ in a sentence.  


\subsection{Other things defined in \textsf{preamble.ins}\label{sec:figures}}
The remaining entries in the \textsf{preamble.ins} file deal with a variety of pre-defined commands that should not
be modified by the user (although you can always define modifications of these commands by using a different command name).

Lines 33-51 define a number of true-false toggles that allow you to generate certain items in the final output.  Each toggle is 
defined here, and set to false by default.  You can use the \la{toggletrue}{} switches in the main source file to turn individual 
items on.

Lines 58-68 define the layout of the page headers and footers; this uses the \textsf{fancyheadings} package, which should be part 
of a complete \LaTeX\ installation.  This is where the \la{shortthesistitle}{} and \la{shortauthor}{} commands are used.

Lines 75-134 define a particular way to deal with figure and table labels, and provide a shorthand notation to insert 
a figure into a document.  There are two sets of command definitions; the value of the \textsf{oldpreamble} toggle determines
which block will be active.\footnote{If you used a version of the \textsf{cmu-thesis.cls} package prior to May 19, 2014, then 
you may have used commands that are no longer supported in the new version of the \textsf{preamble.ins} file. This does not
mean that you need to rework all your source files.  You can simply set the \textsf{oldpreamble} toggle to true in the main
source file to continue to use the older commands.  If you are new to this package, we suggest that you set \textsf{oldpreamble}
to false (the default value).}
 
Lines 112-113 define shorthand commands to cite figures and tables.  If figure 15 has label \textsf{microstructure1}, then you 
can refer to this figure by typing \la{figref}{microstructure1} which produces Fig.~15 in the text.  

If you decide to use the \textsf{epsfig} package for figures,\footnote{You could also use the \textsf{includegraphics} package.}
then the two commands defined on lines 116 and 126 provide convenient ways to do so.  Let's assume that you have a figure
that was created in Adobe Illustrator, and named \textsf{micro1.eps}; the figure is located in the folder set by the \la{graphicspath}{}
command in the main file (see next section).  You can include this figure in your text by using the command:
\begin{verbatim}
	\insertfigure{micro1}{This is the figure caption}
\end{verbatim}
The \lc{insertfigure}{}{} command will use the first argument both as the file name, and as the figure label; you can then use \la{figref}{micro1} to
refer to the figure.  The second command, \lb{insertfigurew}, has an extra argument that allows you to specify the width of the figure:
\begin{verbatim}
	\insertfigurew{micro1}{This is the figure caption}{4.5in}
\end{verbatim}
which will force the width of the figure to be $4.5$ inches, regardless of the size specified in the eps file.

This is just one way to deal with figures.  As you gain more confidence with \LaTeX\ typesetting, you can try to define
new commands that deal with your specific needs, e.g., putting two figures next to each other, putting a figure in the 
margin, etc.  There are also many examples on the web, so you won't have to reinvent the wheel for anything that you
will likely need for your thesis document.





\subsection{Sample master file\label{sec:sample}}
Before we discuss the layout of a typical thesis document, we should explain the difference between the \LaTeX\ \la{input}{}
and \la{include}{} commands.  When you typeset a document, starting from a given .tex file, the \TeX\ program will generate
a series of additional files.  Among those files there is a file with the same name as the source file, and extension .aux (auxiliary).
This is a text file that contains information about citations, figures, tables, section labels, index entries, and so on; the .aux file is how \LaTeX\
keeps track of the overall structure of your document.  When you run the program a second time, the first thing that will happen (after loading
packages) is that the .aux file is read.  If your source consists of multiple files (for instance, a \textsf{main.tex} file and several \textsf{chapter\#.tex}
files), then you can typeset only a portion of the entire document (we'll show below how to do this), but you want the program to still
``know'' all the labels and such from the other portions; this is done by reading all the .aux files.

The difference between \la{input}{} and \la{include}{} is easy to understand:
\begin{itemize}
	\item \la{input}{name.ext} inserts a text file \textsf{name.ext} at the current position in the document.  This could be useful
	to define special commands that are all collected in a single file.  In the current case, this is how the \textsf{preamble.ins}
	file is read.   
	\item \la{include}{name.tex} expects a \LaTeX\ source file as its argument.  This command will also force the 
	typesetting engine to read the corresponding \textsf{name.aux} file, if it exists.  The \la{includeonly}{} command can
	be used to control which files will actually be included in the document.  As an example, let us assume that there is a 
	\textsf{main.tex} source as well as three chapter files, \textsf{chapter01.tex}, \textsf{chapter02.tex}, and \textsf{chapter03.tex}.
	In the main.tex file, you will want to have three \la{include}{} commands as follows:
	\begin{verbatim}
		\include{chapter01}
		\include{chapter02}
		\include{chapter03}
	\end{verbatim}
	Note that the extension .tex is not needed.  If the source is then typeset, all three chapter files and their corresponding .aux files
	will be read and typeset.  If you want to typeset only \textsf{chapter02.tex}, then you must tell the typesetting engine to 
	\la{includeonly}{chapter02}; note that this command must be located in the document preamble, i.e., before the \la{begin}{document}
	command.  In that case, only chapter 2 will be typeset in your output file (typically a PDF file), but the .aux files from the other two
	chapters will still be read.  This allows you to refer to a figure or equation in another chapter (one that was typeset at some earlier
	time, so that its .aux file exists) without actually needing to have that chapter in your output.
\end{itemize}

Next, let's take a look at the \textsf{sample-thesis.tex} file.  Here is an abbreviated version of this file:
\fvset{frame=lines,numbers=left,formatcom=\color{blue},fontsize=\footnotesize}
\VerbatimInput{short-thesis.tex}
Note that comments in \LaTeX\ source files start with the percent character \%; it improves the readability of a 
source file if you intersperse meaningful comments throughout the file.

Line 1 in the document loads the document class file \textsf{cmu-thesis}; this line must be the first one in your file, apart 
from comment lines.  This is followed by the loading of \textsf{package} files; packages are collections of predefined 
\LaTeX\ commands that accomplish a particular typesetting goal.  Line 5 loads a package that replaces the default font (Computer Modern)
by Helvetica, line 6 loads a package that contains commands for using Encapsulated Postscript files, and so on.
The \textsf{sample-thesis.tex} file loads many more packages.  You can browse the \textsf{ctan.org} web site
to get an idea of all available packages, along with manuals for all the commands in each package.

Line 15 loads a package called \textsf{showkeys}, which, when loaded, will print all labels and citations in the margins
of the document; this is useful when editing a lengthy document, to make sure that correct labels and references are
used, but it should be turned off in the final version of the document by commenting out this line.

Line 18 loads the \textsf{hyperref} package, which is used to make the entries in the Table of Contents clickable links inside
the final PDF output file.  In addition, citations and references to equations and sections are also turned into hyperlinks.
Lines 21-28 define the appearance of such links; the user can set the color of links, anchors, urls and so on.  As with
most packages, there are many additional settings and commands; the user should consult the manual pages for this
and other packages to learn how to use them.

Line 31 declares the path for figure files; as discussed in section~\ref{sec:tips}, it is a good idea to keep all your figures
separate from the \LaTeX\ source files, and this command tells the typesetting engine where the figures can be found.

On line 41, you can uncomment the \la{toggletrue}{oldpreamble} command to use the older version of the preamble file;
this is only provided for backwards compatibility for those students who have already created an overview or thesis document,
but wish to use the new preamble file.  

In line 45, the engine is told to load the \textsf{preamble.ins} file, which was discussed in detail in section~\ref{sec:preamble}.

In lines 48-56, you can uncomment any line to include the corresponding item in your output file; so, if you want a title page,
then simply uncomment the line that says \%\la{toggletrue}{titlepage}.

Line 59 declares which source files should be typeset in the output file; in this example, five source files will be included.
To include all source files (which you would need to do for the final typesetting run of your thesis document) you can simply
comment out this line.

The actual document begins at line 62; everything preceding line 62 is known as the \textit{preamble}.  Every \la{begin}{} statement
must have a corresponding \la{end}{} statement, which is found on the last line of the file.  The contents of the document are 
subdivided into three regions: 
\begin{itemize}
	\item \textit{front matter}: this includes the title page, dedication page, preface, table of contents, list of figures, list of tables, list of
	symbols, etc.  Typically, the page numbering of this section uses roman numerals, as defined in line 65.
	\item \textit{main matter}: the main portion of the document consists of multiple chapters, and uses arabic numerals for page numbers.
	The actual page number must be reset to 1 (line 71) before including any chapter files.
	\item \textit{back matter}: this includes the appendices, if any, citations, and index (if one is required; typically, for a thesis
	document, there is no need for an index).  The \la{appendix}{} command must be issued 
	before including any appendix files, to make sure that equation numbers and such are properly formatted (i.e., A-1 instead of 1).
\end{itemize}
The example file shows how to deal with a large document that consists of many sections.  It is good practice to keep each chapter
in its own source file, and to list all chapters, in the correct order, as \la{include}{} commands in the document section of the \textsf{main.tex} file.
Once everything is properly set up, then you will only need to modify the argument of the \la{includeonly}{} command to typeset individual 
sections (chapters) of your document.  You can have many versions of this command commented out, with only the one you currently 
need uncommented, as in this example:
	\begin{verbatim}
		%\includeonly{chapter01}
		%\includeonly{chapter02}
		\includeonly{chapter01,chapter03}
		%\includeonly{chapter03}
	\end{verbatim}
In this example, both chapters 1 and 3 will be typeset, and the .aux file of chapter 2 will be read (provided there is an \la{include}{chapter02}
command in the main document section).


\subsection{How to deal with citations\label{sec:references}}
The file \textsf{references.tex} is needed to define where your bibtex file(s) is/are located.  Open the file with
your favorite editor, and modify the last line of the file (no need to change anything else in this file).  
The repository file will have the entry
\begin{verbatim}
	\bibliography{bibfile}
\end{verbatim}
which indicates that there is a file called \textsf{bibfile.bib} in the current folder.  It is probably a good 
idea to create a separate folder to keep your bibliography files.  Assuming that you have a folder called
\textsf{citations} in your main folder, and you have two bibtex files called \textsf{bib1.bib} and \textsf{bib2.bib},
then you should modify the above line to:
\begin{verbatim}
	\bibliography{citations/bib1,citations/bib2}
\end{verbatim}
Obviously, you can specify a full path name, so that your bibtex files can be located anywhere on your drive.

There is no real need to ever directly edit a bibtex file by hand, since the bibliography programs (e.g., \textsf{BiBDesk})
will take care of all the formatting details.  Just a few comments that will help you avoid common pitfalls:
\begin{enumerate}
	\item \textit{citation labels}: each citation needs to be given a label, that can be used to insert the citation
	in the text using the \textsf{$\backslash$cite$\{\}$} command.  Obviously, citation labels need to be unique.  It is good practice to make the label something that 
	you can easily generate and recognize. While there are a few different ways of doing this, the following method is quite easy
	to apply consistently:  take the last name of the first author, put it in lower case, add the year of publication, and follow
	this by a lowercase letter ``a'', ``b'' etc., to allow for the fact that an author may have more than one paper
	in a given year. so, for the following citation:
	\begin{verbatim}
	C. Phatak, M. Pan, A.K. Petford-Long, S. Hong and M. De Graef, 
	"Magnetic interactions and reversal of artificial square spin ices," 
	New Journal of Physics, vol. 14, 075028 (2012)
	\end{verbatim}
	you would construct the label as ``phatak2012a'' and use \la{cite}{phatak2012a} in the text.
	
	\item \textit{entering author names}: it is good practice to always enter author names in the same way, to avoid potential 
	issues with the formatting of initials and such.  Experience shows that the following approach causes a minimum of problems.
	Consider the citation above; the list of authors should be entered in \textsf{BiBDesk} (or equivalent) as follows:
	\begin{verbatim}
		\author = {Phatak, C. and Pan, M. and Petford-Long, A.K. and Hong, S. and De Graef, M.}
	\end{verbatim}
	Note the use of a comma after each last name, but not after the initials; also note the use of \textit{and} after each author.
	Doing things this way will inform the bibtex program, among other things, that \textit{De} and \textit{Graef} need to be kept 
	together.\footnote{I have, over the years, received regular mail addressed to \textit{Prof.\ De}; surprisingly, the CMU post
	office has always managed to get those letters to the right location.}  Other ways of entering author names may also work, but 
	the method above should not cause any issues.	
	
	\item \textit{how to use capital letters in article/book titles}: the bibtex typesetting program has the annoying habit 
	of automatically lowercasing all uppercase letters in the title of a book or article, except for the first capital.  So, if 
	the title were 
	\begin{verbatim}
		Applications of EBSD to Ni and Al
	\end{verbatim}
	it would be typeset as \textit{Applications of ebsd to ni and al}.  To avoid this, it is necessary to enclose the capital 
	letters between curly braces, as in 
	\begin{verbatim}
		Applications of {EBSD} to {N}i and {A}l
	\end{verbatim}
\end{enumerate}

\section{The front matter\label{sec:frontmatter}}
The front matter section contains a series of items that can be turned on or off (default off) by using the toggle switches in
the main source file.  A typical \textsf{frontmatter.tex} file is provided and reads as follows:
\fvset{frame=lines,numbers=left,formatcom=\color{blue},fontsize=\footnotesize}
\VerbatimInput{frontmatter.tex}
Note that there is one place in the frontmatter file where you will need to change two entries!  Look for the word ``CHANGE''
to find that location.\footnote{Due to some intricacies of \LaTeX, it appears to be difficult to pass the author and date variables from 
the preamble file to the frontmatter file, so for now we'll have to do this manually.}  Then enter your full name and the date at
the appropriate locations and save the file.  Make sure to leave the empty lines between the three commands!

By default, none of the entries in your frontmatter file will appear in the output; you must explicitly turn on those items that you 
wish to typeset.  
\begin{itemize}
	\item \textit{Overview Announcement}: this option produces a properly formatted announcement page for your overview.
	\item \textit{Defense Announcement}:  this option produces a properly formatted announcement page for your defense.
	\item \textit{Title Page}: produces a properly formatted title page for your overview/thesis document.
	\item \textit{Dedication}: add a separate dedication page after the title page.
	\item \textit{Abstract}: an abstract will appear after the dedication page and can be of arbitrary length.
	\item \textit{Acknowledgements}: this is where you thank everyone for whatever help or support they provided; can be of arbitrary length.
	\item \textit{Publication List}: here is where you list all of the publications that were produced during your research period.
	\item \textit{Table of Contents}: includes a hyperlinked Table of Contents in the document.
	\item \textit{List of Figures and Tables}: include separate lists of Figures and Tables.
	\item \textit{List of Symbols}: add your own list of symbols.
\end{itemize}

\section{Useful tips\label{sec:tips}}
Here are a few potentially useful tips as you start preparing your overview/thesis document:
\begin{enumerate}
	\item Create a folder called \textsf{Thesis} (or \textsf{Overview}) somewhere on your hard drive, and in it, a sub-folder called
	\textsf{figures}.  This will allow you to keep figures and \LaTeX\ source separate and clearly organized.  Place all the 
	class files and this README.pdf file inside this top folder, so that you can easily locate it later.
	\item Make sure that all new commands are collected in the \textsf{mycommands.ins} file, so that they are easily maintained.  Create this 
	file first, before you do anything else.
	\item Use a simple name for your master source file, such as \textsf{main.tex} or \textsf{overview.tex}/\textsf{thesis.tex}.  This is the
	only file that you need to typeset, since it inputs/includes all other required files.	  To get started, copy the contents of the \textsf{sample-thesis.tex}
	file into your new master file, so that it does not get overwritten with a new release in the future.
	\item Avoid using figure file names of the type \textsf{Chap5Fig3.eps}; this will only cause problems if you later decide to
	insert a new figure between figures 2 and 3 in chapter 5, or you decide to move this figure to a different chapter.  
	The same goes for chapter source files.  If you are $100\%$ certain
	that the order of your chapters will not ever need to be changed,\footnote{Even if you think that the 
	order of your chapters is correct, don't underestimate the ability of your pesky advisor to decide two weeks before your defense
	that you should change the order.} then you can use filenames of the type \textsf{chapter01.tex},
	\textsf{chapter02.tex}, etc.  It is generally easier to give each chapter file a meaningful name, e.g., \textsf{introduction.tex},
	\textsf{literature.tex}, \textsf{results.tex}; changing the order of chapters in your thesis then simply requires changing the 
	order of the corresponding \la{include}{} commands in the main source file.  Everything else will be automatically renumbered 
	by the typesetting engine.
	\item Avoid using the underscore symbol and spaces in file names; some \LaTeX\ commands may not function properly when you have an
	underscore or a space in a file name.
	\item Make sure you have at least one back-up of all your files.  Better yet, if you know how to use a source code repository, use it to 
	store your \LaTeX\ source code files, so that you can always return to earlier versions.  You could consider your overview document
	to be the first release of your project, and the thesis document the second release.  It is not advisable to store binary files in the 
	repository, so you should put the figures folder in your \textsf{.gitignore} file (or equivalent, if you are using a different type of repository).
	\item To resolve all references, labels, and citations, you need to run the typesetting engine three times for the final production run of 
	your document: the first time, all .aux files are properly created; then you run the bibtex program to generate the bibliography files; then you 
	run the typesetting engine two more times to make sure that all references are resolved.  If you are using the \textsf{TeXShop} program, then
	you can simply typeset using the \textit{pdflatexmk} engine, which does all this automatically.
\end{enumerate}





\end{document}